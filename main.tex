%----------------------------------------------------------------------------------------
%   Доорх хэсгийг өөрчлөх шаардлагагүй
%----------------------------------------------------------------------------------------
%!TEX TS-program = xelatex
%!TEX encoding = UTF-8 Unicode
\documentclass[12pt,A4]{report}

\usepackage{fontspec,xltxtra,xunicode}
\setmainfont[Ligatures=TeX]{Times New Roman}
\setsansfont{Arial}
\usepackage{listings}
\usepackage{xcolor} % for customizing colors

% Define a custom code listing environment
\lstnewenvironment{mycode}[1][]
{
  \lstset{
    language=Python,           % Set the programming language (e.g., Python)
    basicstyle=\ttfamily,      % Set the font style for the code
    numbers=left,              % Display line numbers
    numberstyle=\tiny,         % Style of line numbers
    frame=single,              % Add a frame around the code
    frameround=fttt,           % Set rounded corners for the frame
    breaklines=true,           % Allow code to break across lines
    #1                        % Additional custom options passed by the user
  }
}
{}
% \usepackage[utf8x]{inputenc}
% \usepackage[mongolian]{babel}
%\usepackage{natbib}
\usepackage{geometry}
%\usepackage{fancyheadings} fancyheadings is obsolete: replaced by fancyhdr. JL
\usepackage{fancyhdr}
\usepackage{float}
\usepackage{afterpage}
\usepackage{graphicx}
\usepackage{amsmath,amssymb,amsbsy}
\usepackage{dcolumn,array}
\usepackage{tocloft}
\usepackage{dics}
\usepackage{nomencl}
\usepackage{upgreek}
\newcommand{\argmin}{\arg\!\min}
\usepackage{mathtools}
\usepackage[hidelinks]{hyperref}
\usepackage{bookmark}

\usepackage{algorithm}
\usepackage{algpseudocode}

\usepackage{listings}
\DeclarePairedDelimiter\abs{\lvert}{\rvert}%
\makeatletter
\usepackage{caption}
\captionsetup[table]{belowskip=0.5pt}
\usepackage{subfiles}

\usepackage{url}
\usepackage{listings}
\renewcommand{\lstlistingname}{Код}
\renewcommand{\lstlistlistingname}{\lstlistingname ын жагсаалт}

\usepackage{color}

\definecolor{codegreen}{rgb}{0,0.6,0}
\definecolor{codegray}{rgb}{0.5,0.5,0.5}
\definecolor{codepurple}{rgb}{0.58,0,0.82}
\definecolor{backcolour}{rgb}{0.99,0.99,0.99}
\definecolor{darkgray}{rgb}{0.99,0.103,0.105}
\definecolor{purple}{rgb}{0.128,0,0.128}
 
\lstdefinestyle{mystyle}{
    basicstyle=\ttfamily\small,
    backgroundcolor=\color{backcolour},   
    commentstyle=\color{codegreen},
    keywordstyle=\color{magenta},
    numberstyle=\tiny\color{codegray},
    stringstyle=\color{codepurple},
    %basicstyle=\footnotesize,
    breakatwhitespace=false,         
    breaklines=true,                 
    captionpos=b,                    
    keepspaces=false,                 
    numbers=left,                    
    numbersep=10pt,                  
    showspaces=false,                
    showstringspaces=true,
    showtabs=false,                  
    tabsize=2
}
 
 %define Javascript language
\lstdefinelanguage{JavaScript}{
keywords={typeof, new, true, false, catch, function, return, null, catch, switch, var, if, in, while, do, else, case, break},
keywordstyle=\color{blue}\bfseries,
ndkeywords={class, export, boolean, throw, implements, import, this},
ndkeywordstyle=\color{darkgray}\bfseries,
identifierstyle=\color{black},
sensitive=false,
comment=[l]{//},
morecomment=[s]{/*}{*/},
commentstyle=\color{purple}\ttfamily,
stringstyle=\color{red}\ttfamily,
morestring=[b]',
morestring=[b]"
}
 
\lstset{
language=JavaScript,
extendedchars=true,
basicstyle=\footnotesize\ttfamily,
showstringspaces=false,
showspaces=false,
numbers=left,
numberstyle=\footnotesize,
numbersep=9pt,
tabsize=2,
breaklines=true,
showtabs=false,
captionpos=b
}
 
\lstset{style=mystyle, label=DescriptiveLabel} 


\let\oldabs\abs
\def\abs{\@ifstar{\oldabs}{\oldabs*}}
\makenomenclature
\begin{document}


%----------------------------------------------------------------------------------------
%   Өөрийн мэдээллээ оруулах хэсэг
%----------------------------------------------------------------------------------------

% Дипломийн ажлын сэдэв
\title{Дата аналист}
% Дипломын ажлын англи нэр
\titleEng{Data analyst}
% Өөрийн овог нэрийг бүтнээр нь бичнэ
\author{Даваадавга Адъяабазар}
% Өөрийн овгийн эхний үсэг нэрээ бичнэ
\authorShort{Д. Адъяабазар}
% Удирдагчийн зэрэг цол овгийн эхний үсэг нэр
\supervisor{Ш. Цэнд-Аюуш}
% Хамтарсан удирдагчийн зэрэг цол овгийн эхний үсэг нэр
\cosupervisor{-}

% СиСи дугаар 
\sisiId{20B1NUM0902}
% Их сургуулийн нэр
\university{МОНГОЛ УЛСЫН ИХ СУРГУУЛЬ}
% Бүрэлдэхүүн сургуулийн нэр
\faculty{ХЭРЭГЛЭЭНИЙ ШИНЖЛЭХ УХААН, ИНЖЕНЕРЧЛЭЛИЙН СУРГУУЛЬ}
% Тэнхимийн нэр
\department{МЭДЭЭЛЭЛ, КОМПЬЮТЕРИЙН УХААНЫ ТЭНХИМ}
% Зэргийн нэр
\degreeName{Үйлдвэрлэлийн дадлагын тайлан}
% Суралцаж буй хөтөлбөрийн нэр
\programeName{Мэдээллийн технологи}
% Хэвлэгдсэн газар
\cityName{Улаанбаатар}
% Хэвлэгдсэн огноо
\gradyear{2023 оны 9 сар}


%----------------------------------------------------------------------------------------
%   Доорх хэсгийг өөрчлөх шаардлагагүй
%----------------------------------------------------------------------------------------
%----------------------Нүүр хуудастай хамаатай зүйлс----------------------------
\pagenumbering{roman}
\makefrontpage
\maketitle

\doublespace

% Decleration
% \begin{huge}
% \textbf{Зохиогчийн баталгаа}
% \end{huge} \\ \ \\ 
% \doublespace
% Миний бие \@author \ "\@title" \ сэдэвтэй судалгааны ажлыг гүйцэтгэсэн болохыг зарлаж дараах зүйлсийг баталж байна:
% \begin{itemize}
% \item Ажил нь бүхэлдээ эсвэл ихэнхдээ Монгол Улсын Их Сургуулийн зэрэг горилохоор дэвшүүлсэн болно.
% \item Энэ ажлын аль нэг хэсгийг эсвэл бүхлээр нь ямар нэг их, дээд сургуулийн зэрэг горилохоор оруулж байгаагүй.
% \item Бусдын хийсэн ажлаас хуулбарлаагүй, ашигласан бол ишлэл, зүүлт хийсэн.
% \item Ажлыг би өөрөө (хамтарч) хийсэн ба миний хийсэн ажил, үзүүлсэн дэмжлэгийг дипломын ажилд тодорхой тусгасан. 
% \item Ажилд тусалсан бүх эх сурвалжид талархаж байна. 
% \end{itemize} 
% \ 

% Гарын үсэг: \underline{\hspace{5cm}}

% Огноо: 	\ \ \underline{\hspace{3cm}}

% Гарчгийг автоматаар оруулна
\setcounter{tocdepth}{1}
\tableofcontents

% Зургийн жагсаалтыг автоматаар оруулна
\listoffigures

% Хүснэгтийн жагсаалтыг автоматаар оруулна
\listoftables
% This puts the word "Page" right justified above everything else.
\newpage
%% \addtocontents{lof}{Зураг~\hfill Хуудас \par}
\newpage
%% \addtocontents{lot}{Хүснэгт~\hfill Хуудас \par}

\renewcommand{\cftlabel}{Зураг}


\doublespace
\pagenumbering{arabic}


\begin{abstract}
  Миний бие Д. Адъяабазар нь үйлдвэрлэлийн дадлагын хугацаанд Tableau технологи дээр голчлон ажилласан ба анхан шатнаас ахисан түвшин хүртэлх үйлдэл, боломж, чадваруудад суралцан бэлэн өгөгдөлтэй ажиллан түүнийгээ ашиглан хэрэглэгчид мэдээлэл өгөх дашбоардуудыг хийсэн. Мөн суралцах хугацаандаа docker гэж юу болох хэрхэн ашиглах, зарим query-г хэрхэн бичвэл хурдан ажиллах, Oracle датабааз гэж юу болох зэрэг мэдлэгүүдэд суралцсан.
\end{abstract}


\begin{table}[h]
\caption{Дадлагын төлөвлөгөө}
\begin{tabular}{|p{0.5cm}|p{8cm}|l|l|p{3cm}|}
\hline
\textbf{№} & \textbf{Гүйцэтгэх ажил} & \textbf{Хугацаа} & \textbf{Биелэлт} & \textbf{Дадлагын удирдагчийн үнэлгээ} \\ \hline
1 & Байгууллагын бүтэц зохион байгуулалт, үндсэн үйл ажиллагаатай танилцах & 1 хоног && \\ \hline
2 & Tableau, datagrip программын орчин
тааруулах, байгууллагаас
хэрэглэгчийн лиценз аван бүртгэл
үүсгэх & 1 хоног && \\ \hline
3 & Tableau fundamental 1-3 сэдвийн
агуулгыг бүрэн ойлгож хүн амын тоон
мэдээллийн өгөгдлийг татаж аван
өгөгдлийг цэвэрлэж, sheet үүсгэж
сурах & 5 хоног && \\ \hline
4 & Tableau fundamental 4-6 сэдвийн
агуулгыг бүрэн ойлгож өмнөх
dashboard-ийг сайжруулж дотоодын
нийт бүтээгдэхүүний өгөгдлийг
холбон dashboard угсрах
 & 7 хоног && \\ \hline
5 & Tableau fundamental 7-8 сэдвийн
агуулгыг бүрэн ойлгож өмнөх
dashboard-ийг сайжруулж цалингийн
дундаж хэмжээтэй холбоотой
өгөгдлийг холбон dashboard угсрах & 7 хоног && \\ \hline
6 & Docker программын үйл ажиллагааны
зарчим, командыг бүрэн ойлгож
ашиглаж сурах
 & 2 хоног && \\ \hline
\end{tabular}
\end{table}
\begin{table}[h]
    \begin{tabular}{|p{0.5cm}|p{8cm}|l|l|p{3cm}|}
    \hline
    \textbf{№} & \textbf{Гүйцэтгэх ажил} & \textbf{Хугацаа} & \textbf{Биелэлт} & \textbf{Дадлагын удирдагчийн үнэлгээ} \\ \hline
    
    7 & Docker программын орчныг тааруулж,
    өгөгдлийг docker container-д
    байршуулж эцсийн байдлаар
    хөгжүүлсэн dashboard-г холбох & 2 хоног && \\ \hline
    8 & Oracle foundations associate сертификат
    авахаар бэлдэн, шалгалт өгөх
    х & 5 хоног && \\ \hline
    9 & Төсөлд оролцон “Зөрчлийн бүртгэл
    тоон мэдээлэл”-ийн дашбоардыг
    хамтран гүйцэтгэх & 21 хоног && \\ \hline
    10 & Дадлагын урьдчилсан тайлан бэлтгэж,
    танилцуулах & 7 хоног && \\ \hline
    \end{tabular}
    \end{table}
    

\addcontentsline{toc}{part}{БҮЛГҮҮД}

\chapter{Байгууллагын танилцуулга}
\subfile{chapters/introduction}

\chapter{Ижил системийн судалгаа}
\subfile{chapters/research}

\chapter{Системийн шаардлага}
\subfile{chapters/requirements}

\chapter{Ашиглах технологи}
\subfile{chapters/technologies}

% \chapter{Системийн шинжилгээ}
% \subfile{chapters/analizes}

\chapter{Хэрэгжүүлэлт}
\subfile{chapters/implementation}

\chapter{Дүгнэлт}
\subfile{chapters/conclusion.tex}

\end{document}

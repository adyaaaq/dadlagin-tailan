\section{Tableau}
Tableau нь Мэдээллийн хэлбэр, хэмжээнээc үл хамааран бүх төрлийн өгөгдлийг түргэн хугацаанд боловсруулж, задлан шинжлэх боломжтой BI tool юм. Tableau нь удирдлагын түвшний тайлангаас эхлээд богино хугацаанд боловсруулалт хийх шаардлагатай төрөл бүрийн интерактив тайланг компьютер, интернэт хөтөч, гар утас ашиглан хэдхэн товчлуурын тусламжтайгаар хийх боломжтой.
\subsection{Давуу талууд:}
\begin{itemize}
	\item Өгөгдлүүдийг боловруулахдаа өндөр хурдтайгаар боловсруулдаг
	\item Sql хэл дээр хийхэд хүндрэлтэй query-үүдийг хялбархан хийх боложтой.
	\item Сурахад, энгийн график өрөхөд хялбар.
	\item Олон төрлийн график хийх боломжтой.
	\item Оффисийн программ, датабааз зэргүүдтэй харилцан ажиллах боломжтой.
\end{itemize}


\section{Docker}
Docker гэдэг нь ямар ч програмыг заавал өөрийнхөө компьютер дээр суулгалгүйгээр ажиллуулах боломжтой томоохон шийдэл юм. Жишээ нь Mysql дээр бааз үүсгэн түүн дээрээ ажиллахын тулд заавал Mysql-ийг суулгах хэрэгцээ гарна. Docker нь үүний шийдэл болж бидний компьютер дээр бус Container-д Mysql-ийг суулган түүтэйгээ холбогдон ажиллах боломжийг олгодог. 

Дадлагын хөтөлбөрийн төлөвлөгөөний дагуу Docker-ийг анхан шатна түвшинд суралцан Tableau-ийн дашбоардныхаа датабаазыг docker-д үүсгэж ашигласан.
\subsection{Давуу талууд:}
\begin{itemize}
	\item Ямар ч програмыг заавал өөрийн компьютерт суулгаж ашиглах шаардлагагүй. 
	\item Жижиг хэмтээтэй.
\end{itemize}